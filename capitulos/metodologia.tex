\chapter[Metodologia]{Metodologia}
\label{capMeto}

A metodologia adotada para o desenvolvimento do redesign da interface do sistema DebugandoED e a coleta de percepções de estudantes e profissionais baseia-se em uma abordagem prática e participativa. As etapas do processo envolvem a utilização da ferramenta Figma\footnote{\url{https://www.figma.com/}} para a criação das imagens do redesign da interface, permitindo a prototipagem e visualização das propostas de forma interativa e colaborativa. Além disso, os questionários para coleta de percepções são elaborados e aplicados por meio da plataforma Google Forms\footnote{\url{https://docs.google.com/forms}}.


\section{Desenvolvimento de Protótipos}

Neste estágio, são criadas imagens do redesign da interface do sistema DebugandoED utilizando a ferramenta Figma. A utilização do Figma permite a criação de imagens estáticas das propostas de redesign, possibilitando a visualização e avaliação das diferentes telas e fluxos de navegação. A criação de imagens viabiliza a aplicação prática dos conceitos teóricos de usabilidade, comunicabilidade, acessibilidade, \ac{UX} e \ac{UI}, permitindo a avaliação e refinamento das propostas de design com base na análise das imagens simuladas.

\section{Coleta de Percepções}

Para a coleta de percepções de estudantes e profissionais sobre o redesign da interface, são elaborados questionários estruturados utilizando a plataforma Google Forms. Os questionários abrangem aspectos relevantes relacionados à usabilidade, acessibilidade,  \ac{UX} e \ac{UI} na interface do sistema DebugandoED. A escolha pelo Google Forms como plataforma para a aplicação dos questionários proporciona facilidade na coleta, organização e análise dos dados obtidos, contribuindo para a compreensão das percepções dos participantes.

\subsection{Preocupações Éticas}

Durante a coleta de percepções de estudantes e profissionais sobre o redesign da interface do sistema DebugandoED, são consideradas preocupações éticas relacionadas à privacidade e confidencialidade dos participantes. Os questionários são elaborados de forma anônima, sem a necessidade de identificação dos participantes, garantindo a proteção dos dados coletados. Além disso, é solicitado o consentimento informado dos participantes antes da aplicação dos questionários, esclarecendo os objetivos da pesquisa e garantindo a livre escolha de participação, assegurando a integridade ética da coleta de percepções.
