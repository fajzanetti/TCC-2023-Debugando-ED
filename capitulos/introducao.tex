\chapter[Introdução]{Introdução}

Atualmente um sistema computacional é pensado para solucionar um problema do usuário final, possuindo qualidades como: usabilidade, acessibilidade, comunicabilidade e experiência do usuário. A \ac{GUI} é uma definição da forma de interação entre o usuário do computador e um programa por meio de uma tela ou representação gráfica visual com desenhos, imagens, etc. Geralmente é descrito como a “tela” de um programa. Segundo \citeonline{de2018aspectos}, “A noção de interface gráfica foi divulgada com o Apple Macintosh. O objetivo era trazer ao grande público um sistema de manipulação de informações de fácil manuseio em analogia com os objetos do nosso dia a dia (pastas, arquivos, lixeiras)”.
Pode-se dizer que é a interface gráfica que faz a \ac{IHC}.

A área de \ac{IHC} pode ser definida como:
 \begin{citacao}
“A área de Interação Humano-Computador (\acs{IHC}) se dedica a estudar os fenômenos de comunicação entre pessoas e sistemas computacionais que está na interseção das ciências da computação e informação e ciências sociais e comportamentais e envolve todos os aspectos relacionados com a interação entre usuários e sistemas . A pesquisa em \acs{IHC} tem por objetivo fornecer explicações e previsões para fenômenos de interação usuário-sistema e resultados práticos para o projeto da interação.” \cite{sbcihc}.
\end{citacao}


Em \acs{IHC}, a avaliação de uma interface de sistema como eficaz requer a consideração de critérios específicos. Os critérios de qualidade de uso destacam características cruciais da interação e da interface, garantindo sua adequação aos objetivos pretendidos durante a utilização do sistema., sendo \cite{barbosa2010interaccao}:
\begin{itemize}
    \item \textbf{usabilidade} - o estágio em que um produto é usado por usuários específicos para conseguir objetivos específicos, eficiência e satisfação em um contexto de uso específico;
    \item \textbf{experiência do usuário} - as percepções e respostas de uma pessoa que resultam do uso ou da expectativa de uso de um produto, sistema ou serviço;
    \item \textbf{acessibilidade} - condição para utilização, com segurança e autonomia, total ou assistida, de sistemas e meios de comunicação e informação, por pessoa portadora de deficiência ou com mobilidade reduzida e
    \item \textbf{comunicabilidade} - responsabilidade do designer comunicar ao usuário suas intenções de design e a lógica que rege o comportamento da interface.
\end{itemize}

Hoje em dia, mesmo com a grande utilização dos termos \ac{UI} e \ac{UX}, muitos que trabalham nessa área ainda confundem os dois termos, mesmo sendo bem distintos. \acf{UI} relaciona-se com algo bem mais objetivo e controlável. Ou seja, é a interface do usuário em que estão o layout do sistema, botões, ícones, imagens, etc. Nesse item, os conceitos de usabilidade, acessibilidade e comunicabilidade são postos em prática, todos esses detalhes fazem parte do \ac{UID}, sendo responsável principalmente pela criação de interfaces funcionais, as quais permitem que usuário navegue intuitivamente por toda sua jornada. 

Já a \acf{UX} relaciona-se a algo mais subjetivo. Isto é, por mais que um designer ou web designer se esforce, ele não tem 100\% de controle sobre aquilo que as pessoas vão sentir quando experimentarem um produto que ele projetou. Dessa forma, o principal papel do \ac{UXD} é se preocupar com cada etapa com a qual o usuário interage com o produto ou serviço e fazer com que essa interação ocorra de modo mais tranquilo possível.

O presente trabalho apresenta um levantamento bibliográfico sobre os temas relacionados a concepção e desenvolvimento de interfaces gráficas para páginas e sistemas web e aplicações móveis, bem como \acs{UI} e \acs{UX}. Neste sentido, pretende-se aplicar os métodos e conceitos na prática, abordando sua utilização nas interfaces dos dias atuais. Além disso, serão aplicadas e verificadas as mesmas metodologias e conceitos em um estudo de caso real, abordando uma plataforma didática web desenvolvida por um programador sem os conhecimentos prévios de \acs{IHC}, \acs{UI} e \acs{UX}.


%\begin{itemize}
%    \item Usabilidade: Facilidade de aprendizado e uso da interface, bem como a satisfação do usuário em decorrência desse uso.
%    \item Experiência do usuário: Emoções e sentimentos dos usuários em relação ao sistema computacional.
%    \item Acessibilidade: Remoção de barreiras que impedem mais usuários de serem capazes de acessar a interface do sistema e interagirem com ele.
%    \item Comunicabilidade: Responsabilidade do designer comunicar ao usuário suas intenções de design e a lógica que rege o comportamento da interface.
%\end{itemize}

%\section{Motivação}
%Introduza o leitor ao assunto, descreva os fatores motivadores para o desenvolvimento do seu trabalho. Descreva brevemente o estado da arte e indique os problemas que ainda não foram resolvidos. Faça um gancho para a próxima sub-seção em que você descreve os  objetivos do seu trabalho. 

\section{Objetivos}

\subsection{Objetivo Geral}
Este trabalho tem como objetivo realizar um levantamento sobre temas atuais de \ac{IHC} na área de design de interfaces, com foco em \ac{UX} e \ac{UI}. O estudo será aplicado em um caso prático.

\subsection{Objetivos Específicos}
\begin{itemize}
    \item Comparar interfaces antigas e atuais de um mesmo sistema, aplicação ou serviço digital já estabelecido no mercado, a fim de analisar e contextualizar as principais mudanças ao longo do tempo.
    \item Realizar uma análise prática do sistema desenvolvido por \citeonline{debugandoedsbsi}, aplicando os conceitos de \ac{UX} e \ac{UI}.
\end{itemize}


%Descreva claramente os desafios que o tema propõe e quais os  objetivos que se pretende alcançar. Se o tema for muito abrangente, descreva os objetivos em termos de "objetivo geral" e  "objetivos específicos". Cuidado com objetivos como "desenvolver um sistema...."; "explorar um método ...".Esses objetivos são triviais, ou seja, uma vez desenvolvido o sistema ou explorado o método, independente dos resultados, o objetivo foi atingido. Prefira verbos como: "contribuir", "analisar", "investigar", "comparar". Os membros da banca ao lerem essa seção farão o seguinte questionamento: Algum conhecimento novo para a humanidade foi produzido?


%\section{Hipótese}
%Descreva claramente quais são as hipóteses da sua pesquisa (Uma hipótese é uma suposição para a solução do problema que você pretende desenvolver). Indique quais perguntas estão associadas a sua hipótese. Lembre-se que as hipóteses deverão ser comprovadas via os experimentos que serão descritos no capítulo \ref{experimentos}.

%\section{Contribuições}
%Liste as contribuições do seu trabalho. Lembre-se que publicações não são contribuições científicas do seu trabalho. Haverá uma seção específica com esse fim.

\section{Organização da Monografia}

O \autoref{capFund} aborda toda a fundamentação teórica, apresentando os conceitos necessários para auxiliar na compreensão do trabalho. No \autoref{capExperimentos}, são apresentados os experimentos, análises e resultados, aplicando os conceitos discutidos anteriormente e descrevendo os resultados das pesquisas de opinião sobre os experimentos realizados. Finalmente, o \autoref{capConclusao} engloba as conclusões, juntamente com as principais contribuições obtidas neste trabalho.

%Descreva como a sua monografia está organizada, descrevendo brevemente o conteúdo de cada capítulo.