\chapter[Introdução]{Introdução}
\label{capIntro}

Atualmente um sistema computacional é pensado para solucionar um problema do usuário final, possuindo qualidades como: usabilidade, acessibilidade, comunicabilidade e experiência do usuário. A \ac{GUI} é uma definição da forma de interação entre o usuário do computador e um programa por meio de uma tela ou representação gráfica visual com desenhos, imagens, etc. Geralmente é descrito como a “tela” de um programa. Segundo \citeonline{de2018aspectos}, “A noção de interface gráfica foi divulgada com o Apple Macintosh. O objetivo era trazer ao grande público um sistema de manipulação de informações de fácil manuseio em analogia com os objetos do nosso dia a dia (pastas, arquivos, lixeiras)”.
Pode-se dizer que é a interface gráfica que faz a \ac{IHC}.

A área de \ac{IHC} pode ser definida como:
 \begin{citacao}
“A área de Interação Humano-Computador (\acs{IHC}) se dedica a estudar os fenômenos de comunicação entre pessoas e sistemas computacionais que está na interseção das ciências da computação e informação e ciências sociais e comportamentais e envolve todos os aspectos relacionados com a interação entre usuários e sistemas . A pesquisa em \acs{IHC} tem por objetivo fornecer explicações e previsões para fenômenos de interação usuário-sistema e resultados práticos para o projeto da interação.” \cite{sbcihc}.
\end{citacao}


Em \acs{IHC}, a avaliação de uma interface de sistema como eficaz requer a consideração de critérios específicos. Os critérios de qualidade de uso destacam características cruciais da interação e da interface, garantindo sua adequação aos objetivos pretendidos durante a utilização do sistema., sendo \cite{barbosa2010interaccao}:
\begin{itemize}
    \item \textbf{usabilidade} - o estágio em que um produto é usado por usuários específicos para conseguir objetivos específicos, eficiência e satisfação em um contexto de uso específico;
    \item \textbf{experiência do usuário} - as percepções e respostas de uma pessoa que resultam do uso ou da expectativa de uso de um produto, sistema ou serviço;
    \item \textbf{acessibilidade} - condição para utilização, com segurança e autonomia, total ou assistida, de sistemas e meios de comunicação e informação, por pessoa portadora de deficiência ou com mobilidade reduzida e
    \item \textbf{comunicabilidade} - responsabilidade do designer comunicar ao usuário suas intenções de design e a lógica que rege o comportamento da interface.
\end{itemize}

Hoje em dia, mesmo com a grande utilização dos termos \ac{UI} e \ac{UX}, muitos que trabalham nessa área ainda confundem os dois termos, mesmo sendo bem distintos. \acf{UI} relaciona-se com algo bem mais objetivo e controlável. Ou seja, é a interface do usuário em que estão o layout do sistema, botões, ícones, imagens, etc. Nesse item, os conceitos de usabilidade, acessibilidade e comunicabilidade são postos em prática, todos esses detalhes fazem parte do \ac{UID}, sendo responsável principalmente pela criação de interfaces funcionais, as quais permitem que usuário navegue intuitivamente por toda sua jornada. 

Já a \acf{UX} relaciona-se a algo mais subjetivo. Isto é, por mais que um designer ou web designer se esforce, ele não tem 100\% de controle sobre aquilo que as pessoas vão sentir quando experimentarem um produto que ele projetou. Dessa forma, o principal papel do \ac{UXD} é se preocupar com cada etapa com a qual o usuário interage com o produto ou serviço e fazer com que essa interação ocorra de modo mais tranquilo possível.

O presente trabalho apresenta um levantamento bibliográfico sobre os temas relacionados a concepção e desenvolvimento de interfaces gráficas para páginas e sistemas web e aplicações móveis, bem como \acs{UI} e \acs{UX}. Neste sentido, pretende-se aplicar os métodos e conceitos na prática, abordando sua utilização nas interfaces dos dias atuais. Além disso, serão aplicadas e verificadas as mesmas metodologias e conceitos em um estudo de caso real, abordando uma plataforma didática web desenvolvida por um programador sem os conhecimentos prévios de \acs{IHC}, \acs{UI} e \acs{UX}.


\section{Objetivos}

\subsection{Objetivo Geral}
Este estudo tem como propósito principal realizar o aprimoramento da interface do sistema DebugandoED\footnote{\url{https://debugandoed.facom.ufu.br/}} desenvolvido por \citeonline{debugandoedsbsi}, aplicando conceitos fundamentais de usabilidade, comunicabilidade, acessibilidade, \ac{UX} e \ac{UI}. Adicionalmente, busca-se investigar as percepções de estudantes e profissionais em relação à proposta de redesign, com o intuito de avaliar a eficácia das alterações propostas.

\subsection{Objetivos Específicos}
\begin{itemize}
    \item Realizar o levantamento teórico sobre os conceitos de usabilidade, comunicabilidade, acessibilidade, \ac{UX} e \ac{UI}, aplicáveis ao redesign de interfaces de sistemas.
    \item Aplicar os conceitos teóricos na prática, efetuando o redesign da interface do sistema DebugandoED, com foco na otimização dos aspectos de usabilidade, acessibilidade e estética.
    \item Conduzir um estudo de caso real, envolvendo a aplicação da nova interface do sistema DebugandoED, e coletar as percepções de estudantes e profissionais por meio de questionários.
    \item Analisar as percepções coletadas, identificando pontos positivos e áreas passíveis de melhoria na proposta de redesign da interface, com o intuito de validar a aplicação prática dos conceitos de usabilidade, \ac{UX} e \ac{UI}.
    \item Analisar as percepções coletadas, identificando pontos fortes e áreas de melhoria na proposta de redesign da interface, a fim de validar a aplicação prática dos conceitos de usabilidade, \ac{UX} e \ac{UI}.

\end{itemize}

Esses objetivos visam contribuir para o avanço no entendimento das práticas contemporâneas de design de interfaces, evidenciando a importância da usabilidade e estética na experiência do usuário em sistemas de informação.

\section{Organização da Monografia}

Este trabalho está organizado em cinco capítulos. O \autoref{capIntro} apresenta a introdução, contextualizando o tema, os objetivos gerais e específicos, e a organização da monografia. O \autoref{capMeto} descreve a metodologia utilizada para o desenvolvimento do redesign da interface e a coleta de percepções dos participantes. O \autoref{capFund} aborda a fundamentação teórica, incluindo os principais conceitos que contribuirão para o entendimento do trabalho. O \autoref{capExperimentos} descreve os experimentos, análises e resultados, aplicando os conceitos discutidos anteriormente e descrevendo os resultados das pesquisas de opinião sobre os experimentos realizados. Por fim, o \autoref{capConclusao} engloba as conclusões, juntamente com as principais contribuições obtidas neste trabalho.
