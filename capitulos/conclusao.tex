\chapter[Conclusão]{Conclusão}
\label{capConclusao}

Este trabalho teve como principal objetivo aprofundar a compreensão de elementos fundamentais no \textit{design} de interfaces de sistemas, abrangendo temas como usabilidade, comunicabilidade, acessibilidade, experiencia de usuário, interface de usuário e técnicas de avaliação da usabilidade. O estudo iniciou-se com uma revisão teórica aprofundada, onde conceitos propostos por renomados profissionais da área, como Jakob Nielsen, foram discutidos em conjunção com princípios da psicologia da forma, Gestalt, e diretrizes contemporâneas de design, como o Material Design.

A fase empírica do estudo envolveu a aplicação prática de regras e diretrizes extraídas da literatura. Através da construção de protótipos e da implementação de mudanças em sistemas existentes, procurou-se otimizar aspectos como usabilidade e acessibilidade, sempre levando em conta os princípios de comunicabilidade e estética. Para validar as implementações, foi realizado um experimento envolvendo questionários distribuídos a um grupo diversificado de participantes, composto tanto por estudantes quanto por profissionais da área de desenvolvimento.

Após analisar as respostas coletadas por meio de um questionário aplicado a uma diversificada amostra composta por 44 participantes, dos quais 33 eram estudantes universitários e 11 profissionais especializados na área, foi possível identificar percepções ricas e multifacetadas relacionadas à usabilidade, \ac{UI} e \ac{UX} em interfaces de sistemas.

Pôde-se perceber que as alterações propostas, baseadas nos conceitos e princípios estudados, acabaram sendo a opção predominante entre a maioria dos participantes de ambas as categorias, sejam eles estudantes ou profissionais, em todas as interfaces examinadas: Login, Cadastro, Principal e Conteúdo. Esta constatação reforça a ideia de que, independentemente da formação ou experiência prévia do usuário, existem princípios de usabilidade, \ac{UI} e \ac{UX} que possuem uma natureza quase universal, sendo identificáveis e valorizados por uma vasta diversidade de usuários.

Entretanto, cabe ressaltar que apesar das propostas de modificações terem sido amplamente escolhidas, a proposta original não foi completamente desconsiderada. Alguns de seus elementos, como rótulos elucidativos e a harmonia estética, foram positivamente destacados por vários respondentes, em particular quando se tratou da interface de Cadastro. Tal observação evidencia a relevância das heurísticas estabelecidas por Jakob Nielsen (\ref{Heurísticas de Jakob Nielsen}), bem como dos  Princípios de Gestalt (\ref{Princípios de Gestalt}), que dão ênfase à clareza, consistência e à coesão estética global de um design.

Os resultados obtidos indicaram uma série de percepções cruciais sobre a interação entre diferentes aspectos do \textit{design} de interfaces. As respostas dos participantes proporcionaram uma visão clara dos pontos de concordância e discordância entre a teoria e a experiência prática. Além disso, foi possível discernir nuances nas opiniões entre os grupos de estudantes e profissionais, fornecendo uma perspectiva multidimensional sobre o tema.

Em suma, o presente trabalho contribui significativamente para a literatura na área, fornecendo orientações estratégicas e práticas para designers e desenvolvedores. Também enfatiza a importância de avaliar e integrar diversos conceitos e diretrizes no \textit{design} de interfaces para proporcionar uma experiência de usuário excepcional e otimizada.	

%Após uma criteriosa análise das respostas coletadas por meio de um questionário aplicado a uma diversificada amostra composta por 44 participantes, dos quais 33 eram estudantes universitários e 11 profissionais especializados na área, foi possível identificar percepções ricas e multifacetadas relacionadas à usabilidade, \ac{UI} e \ac{UX} em interfaces de sistemas.

%De forma reiterada, a \textbf{"Imagem B"} emergiu como a opção predominante entre a maioria dos participantes de ambas as categorias, sejam eles estudantes ou profissionais, em todas as interfaces examinadas: Login, Cadastro, Principal e Conteúdo. Esta constatação reforça a ideia de que, independentemente da formação ou experiência prévia do usuário, existem princípios basilares de usabilidade, \ac{UI} e \ac{UX} que possuem uma natureza quase universal, sendo identificáveis e valorizados por uma vasta diversidade de usuários.

%Entretanto, cabe ressaltar um ponto de interesse: ainda que a \textbf{"Imagem B"} tenha sido amplamente favorecida, a \textbf{"Imagem A"} não foi completamente desconsiderada. Alguns de seus elementos, como rótulos elucidativos e a harmonia estética, foram positivamente destacados por vários respondentes, em particular quando se tratou da interface de Cadastro. Tal observação evidencia a relevância das heurísticas estabelecidas por Jakob Nielsen (\ref{Heurísticas de Jakob Nielsen}), bem como dos  Princípios de Gestalt (\ref{Princípios de Gestalt}), que dão ênfase à clareza, consistência e à coesão estética global de um design.

\section{Principais Contribuições}
\begin{enumerate}
    \item \textbf{Usabilidade e Design:} Os resultados que se obtive, claramente enfatizam a importância de se desenvolver uma interface que seja ao mesmo tempo intuitiva e bem estruturada, com um visual que chame a atenção e que não deixe a desejar. Isso reforça as ideias e princípios de design e usabilidade que foram discutidos anteriormente neste trabalho, como é o caso de Jakob Nielsen.
    \item \textbf{Técnicas de Avaliação de Usabilidade:} Neste trabalho, adotou-se uma abordagem comparativa bem direta, colocando lado a lado duas imagens diferentes e pedindo a opinião dos participantes sobre elas.
    \item \textbf{Comunicabilidade e Acessibilidade:} As respostas sobre a proposta original ressalta a importância de etiquetas claras e mensagens informativas, especialmente para usuários com mais idade, abordando a necessidade de considerar a diversidade dos usuários.
    \item \textbf{UI e UX:} A predominância das modificações propostas sugere que uma abordagem mais funcional e eficaz na criação de interfaces tende a ser mais bem recebida. Ao mesmo tempo, os elementos estéticos também são vitais para uma experiência de usuário abrangente.
    \item \textbf{Princípios de Gestalt:} A análise indica que a organização clara e a hierarquia visual são fundamentais para a experiência do usuário, alinhando-se com os princípios de Gestalt sobre a percepção.
    \item \textbf{Heurísticas de Jakob Nielsen:} As respostas, especialmente em relação à clareza, erros e organização das interfaces, refletem a relevância das heurísticas de Nielsen na avaliação da usabilidade.
\end{enumerate}
 
%Em suma, este estudo contribuiu para uma compreensão mais profunda de como diferentes grupos demográficos percebem a usabilidade e o design de interfaces de sistemas. Também oferece percepções valiosos para designers e desenvolvedores sobre o que considerar ao projetar interfaces de usuário para proporcionar uma experiência superior.

Em uma análise conclusiva, este trabalho de pesquisa proporcionou um aprofundamento significativo na compreensão de como distintos grupos demográficos interpretam e valorizam a usabilidade e a estética de interfaces de sistemas. A relevância dos achados desta pesquisa não se restringe apenas ao entendimento teórico, mas também possui aplicabilidade prática, podendo ser diretamente empregados no aperfeiçoamento dos sistemas avaliados. Além disso, fornece orientações estratégicas para designers e desenvolvedores, delineando aspectos cruciais a serem levados em conta ao projetar interfaces de usuário com o objetivo de garantir uma experiência de uso excepcional e otimizada.

Além do exposto, o presente trabalho foi parcialmente publicado nos anais do XVI Workshop de Teses e Dissertações em Ciência da Computação (WTDCC) no ano de 2022 com o título: “Uso de Heurísticas e Princípios para Análise de Interface e Experiência de Usuário” \cite{Zanetti2022}.

\section{Trabalhos Futuros}

Como trabalhos futuros, pretende-se criar a proposta para todas as telas e aplicar as contribuições propostas por este trabalho na Plataforma DebugandoED. Além disso, pretende-se publicar os resultados obtidos em um evento nacional, uma vez que o trabalho foi apenas parcialmente publicado, sem o resultados.
