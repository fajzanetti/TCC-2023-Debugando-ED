\chapter[Conclusão]{Conclusão}
\label{capConclusao}

Este trabalho teve como principal objetivo aprofundar a compreensão de elementos fundamentais no \textit{design} de interfaces de sistemas, abrangendo temas como usabilidade, comunicabilidade, acessibilidade, experiencia de usuário, interface de usuário e técnicas de avaliação da usabilidade. O estudo iniciou-se com uma revisão teórica aprofundada, onde conceitos propostos por renomados profissionais da área, como Jakob Nielsen, foram discutidos em conjunção com princípios da psicologia da forma, Gestalt, e diretrizes contemporâneas de design, como o Material Design.

A fase empírica do estudo envolveu a aplicação prática de regras e diretrizes extraídas da literatura. Através da construção de protótipos no sistema DebugandoED, procurou-se otimizar aspectos como usabilidade e acessibilidade, sempre levando em conta os princípios de comunicabilidade e estética. Para validar as implementações, foi realizado um experimento envolvendo questionários distribuídos a um grupo diversificado de participantes, composto tanto por estudantes quanto por profissionais da área de desenvolvimento.

Após analisar as respostas coletadas por meio de um questionário aplicado a uma diversificada amostra composta por 44 participantes, dos quais 33 eram estudantes universitários e 11 profissionais especializados na área, foi possível identificar percepções ricas e multifacetadas relacionadas à usabilidade, \ac{UI} e \ac{UX} em interfaces de sistemas.

Pôde-se perceber que as alterações propostas, baseadas nos conceitos e princípios estudados, acabaram sendo a opção predominante entre a maioria dos participantes de ambas as categorias, sejam eles estudantes ou profissionais, em todas as interfaces examinadas: Login, Cadastro, Principal e Conteúdo. Esta constatação reforça a ideia de que, independentemente da formação ou experiência prévia do usuário, existem princípios de usabilidade, \ac{UI} e \ac{UX} que possuem uma natureza quase universal, sendo identificáveis e valorizados por uma vasta diversidade de usuários.

Entretanto, cabe ressaltar que apesar das propostas de modificações terem sido amplamente escolhidas, a proposta original não foi completamente desconsiderada. Alguns de seus elementos, como rótulos elucidativos e a harmonia estética, foram positivamente destacados por vários respondentes, em particular quando se tratou da interface de Cadastro. Tal observação evidencia a relevância das heurísticas estabelecidas por Jakob Nielsen (\ref{Heurísticas de Jakob Nielsen}), bem como dos  Princípios de Gestalt (\ref{Princípios de Gestalt}), que dão ênfase à clareza, consistência e à coesão estética global de um design.

Os resultados obtidos indicaram uma série de percepções cruciais sobre a interação entre diferentes aspectos do \textit{design} de interfaces. As respostas dos participantes proporcionaram uma visão clara dos pontos de concordância e discordância entre a teoria e a experiência prática. Além disso, foi possível discernir nuances nas opiniões entre os grupos de estudantes e profissionais, fornecendo uma perspectiva multidimensional sobre o tema.

Em suma, o presente trabalho contribui para uma melhoria do sistema DebugandoED, fornecendo orientações estratégicas e práticas para designers e desenvolvedores. Também enfatiza a importância de avaliar e integrar diversos conceitos e diretrizes no \textit{design} de interfaces para proporcionar uma experiência de usuário melhorada.	

\section{Principais Contribuições}
\begin{enumerate}
    \item \textbf{Usabilidade e Design:} Os resultados que, claramente enfatizam a importância de se desenvolver uma interface que seja ao mesmo tempo intuitiva e bem estruturada, com um visual que chame a atenção e que não deixe a desejar. Isso reforça as ideias e princípios de design e usabilidade que foram discutidos anteriormente neste trabalho, como é o caso de Jakob Nielsen.
    \item \textbf{Técnicas de Avaliação de Usabilidade:} Neste trabalho, adotou-se uma abordagem comparativa bem direta, colocando lado a lado duas imagens diferentes e pedindo a opinião dos participantes sobre elas.
    \item \textbf{Comunicabilidade e Acessibilidade:} As respostas sobre a proposta original ressalta a importância de etiquetas claras e mensagens informativas, especialmente para usuários com mais idade, abordando a necessidade de considerar a diversidade dos usuários.
    \item \textbf{UI e UX:} A predominância das modificações propostas sugere que uma abordagem mais funcional e eficaz na criação de interfaces tende a ser mais bem recebida. Ao mesmo tempo, os elementos estéticos também são vitais para uma experiência de usuário abrangente.
    \item \textbf{Princípios de Gestalt:} A análise indica que a organização clara e a hierarquia visual são fundamentais para a experiência do usuário, alinhando-se com os princípios de Gestalt sobre a percepção.
    \item \textbf{Heurísticas de Jakob Nielsen:} As respostas, especialmente em relação à clareza, erros e organização das interfaces, refletem a relevância das heurísticas de Nielsen na avaliação da usabilidade.
\end{enumerate}
 
Em uma análise conclusiva, este trabalho de pesquisa proporcionou um aprofundamento significativo na compreensão de como distintos grupos demográficos interpretam e valorizam a usabilidade e a estética de interfaces de sistemas. A relevância dos achados desta pesquisa não se restringe apenas ao entendimento teórico, mas também possui aplicabilidade prática, podendo ser diretamente empregados no aperfeiçoamento dos sistemas avaliados. Além disso, fornece orientações estratégicas para designers e desenvolvedores, delineando aspectos cruciais a serem levados em conta ao projetar interfaces de usuário com o objetivo de garantir uma experiência de uso melhorada.

Além do exposto, o presente trabalho foi parcialmente publicado nos anais do \ac{WTDCC} no ano de 2022 com o título: “Uso de Heurísticas e Princípios para Análise de Interface e Experiência de Usuário“ \cite{Zanetti2022}.

\section{Limitações}

A condução deste estudo proporcionou uma compreensão mais aprofundada dos elementos essenciais no design de interfaces de sistemas, abordando conceitos como usabilidade, comunicabilidade, acessibilidade, \ac{UX} e \ac{UI} e técnicas de avaliação da usabilidade. Contudo, algumas limitações devem ser reconhecidas para contextualizar os resultados obtidos.

Em relação à coleta de percepções, a quantidade de participantes no questionário, embora diversificada, pode apresentar limitações quanto à representatividade estatística. Apesar de os 44 participantes oferecerem uma variedade de perspectivas, é importante considerar que uma amostra maior poderia proporcionar uma análise  mais robusta.

Outra limitação está associada à implementação real das propostas de redesign. O estudo baseou-se em protótipos visuais para teste de percepções, sem uma implementação prática completa. A ausência dessa implementação real pode impactar a compreensão total do impacto das modificações propostas na experiência do usuário durante a interação com o sistema.

Adicionalmente, a diversidade dos participantes, embora enriquecedora, pode apresentar desafios na generalização dos resultados. Diferenças significativas nas experiências e conhecimentos dos participantes podem influenciar suas percepções, sendo importante considerar essas nuances ao interpretar os dados.

Por fim, é relevante destacar que, embora a proposta original tenha recebido críticas e sugestões de melhoria, alguns elementos da interface original foram elogiados pelos participantes. Isso ressalta a subjetividade do design e a importância de considerar diversas perspectivas ao buscar aprimoramentos.

Essas limitações, embora reconhecidas, não invalidam as valiosas contribuições deste estudo. Os resultados fornecem percepções e direcionamentos para futuras pesquisas e implementações práticas no desenvolvimento de interfaces de sistemas, contribuindo para o avanço contínuo na área de design e usabilidade.

\section{Trabalhos Futuros}

Considerando as limitações identificadas neste estudo, algumas sugestões para trabalhos futuros emergem com o intuito de fortalecer e ampliar as contribuições deste trabalho.

Uma das possibilidades é expandir a quantidade de participantes na coleta de percepções, buscando uma amostra ainda mais diversificada e representativa. Isso proporcionaria uma análise mais robusta e possibilitaria a generalização dos resultados para um público mais amplo.

Além disso, a implementação real das propostas de redesign é uma etapa crucial para avaliar o impacto prático das modificações na experiência do usuário. Portanto, futuros trabalhos podem focar na execução completa das propostas, permitindo uma avaliação mais aprofundada da eficácia das alterações na interação real dos usuários com o sistema.

Outra abordagem promissora seria investigar mais profundamente as nuances nas percepções dos participantes, considerando as diferenças significativas em suas experiências e conhecimentos. Isso poderia levar a uma compreensão mais refinada das preferências e expectativas de diferentes grupos demográficos, contribuindo para um design mais inclusivo e personalizado.

Além disso, dada a subjetividade inerente ao design, futuros estudos podem explorar a integração de feedbacks positivos e elementos elogiados da interface original. Isso pode enriquecer ainda mais a compreensão das preferências dos usuários e orientar o desenvolvimento de interfaces que equilibrem inovação com elementos já apreciados.

Em síntese, os trabalhos futuros podem se concentrar em superar as limitações identificadas, aprimorando a abrangência e aplicabilidade das descobertas deste estudo e contribuindo para o constante avanço na área de design e usabilidade de interfaces de sistemas.