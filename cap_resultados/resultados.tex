\chapter[Resultados]{Resultados e Análises}

% Em análise.. pensando em juntar no capítulo anterior!

Neste capítulo, as respostas obtidas nos questionários da pesquisa conduzida neste estudo serão apresentados. O objetivo principal deste capítulo é fornecer uma visão abrangente das percepções e opiniões coletadas por duas abordagens distintas, em relação à usabilidade, design e experiência do usuário em interfaces de sistema. As diferenças demográficas e de formação educacional entre esses dois grupos proporcionaram uma visão diversificada e enriquecedora das opiniões e perspectivas dos usuários.

A seção \ref{Pesquisa_de_Opinião_com_Estudantes}, concentra-se nas respostas e percepções compartilhados por um grupo representativo de estudantes universitários. Esses jovens representam a próxima geração de profissionais e pesquisadores nesta área de estudo. Suas opiniões e perspectivas podem oferecer uma visão valiosa sobre como a pesquisa e o conhecimento estão sendo percebidos pelos futuros atores do campo.

A seção \ref{Pesquisa_de_Opinião_com_Profissionais}, concentra-se nas respostas e percepções coletadas junto a um grupo seleto de especialistas e profissionais da área. Esses participantes foram escolhidos por sua vasta experiência e conhecimento no domínio de estudo. Suas opiniões são fundamentais para oferecer uma perspectiva madura e experiente sobre as questões abordadas nesta pesquisa.

\section{Resultado - Pesquisa de Opinião com Estudantes}
\label{Pesquisa_de_Opinião_com_Estudantes}
A pesquisa contou com a participação de um grupo diversificado de respondentes, incluindo indivíduos de diferentes gêneros, faixas etárias, níveis de escolaridade e profissões. Os participantes variaram desde estudantes do ensino médio até aqueles com pós-doutorado e experiência em áreas como tecnologia, atendimento ao cliente, e ciências biológicas. Além disso, houve uma distribuição de níveis de experiência, com participantes desde juniores até sêniores em suas carreiras.

Além de suas características demográficas e educacionais, os participantes também foram questionados sobre seu conhecimento em \ac{UI} e \ac{UX}, com algumas pessoas afirmando ter um entendimento desses conceitos, enquanto outras não estavam familiarizadas.

Esta diversidade de perfil dos participantes enriqueceu a pesquisa, fornecendo uma variedade de perspectivas e experiências que contribuíram para uma análise abrangente das telas de interface do usuário e da experiência do usuário. A seguir, apresentaremos as conclusões e \textit{insights} obtidos com base nas respostas e avaliações desses participantes.

\begin{enumerate}
    \item \textbf{Análise das Telas de Login}: Durante esta pesquisa, avaliamos as telas de login "Imagem A" e "Imagem B" em vários aspectos de usabilidade e design. Os resultados revelaram uma clara preferência pela "Imagem B" em categorias cruciais, como \textit{layout} atrativo, experiência intuitiva, organização de informações clara e distribuição eficiente de campos de entrada. No entanto, a "Imagem A" recebeu elogios por sua mensagem de erro e rótulos de botões informativos. Alguns participantes sugeriram combinar as duas telas, destacando pontos fortes de cada uma. Em resumo, enquanto a "Imagem B" se destacou como a escolha preferida na maioria das categorias, a "Imagem A" demonstrou pontos fortes específicos que merecem consideração na otimização da experiência do usuário.
    
    \item \textbf{Análise das Telas de Cadastro}: Na análise comparativa entre as "Imagem A" e "Imagem B" para a tela de cadastro, observamos uma preferência geral pelos elementos da "Imagem B" em diversas categorias, incluindo \textit{layout}, organização, distribuição de campos de entrada, mensagens de erro e conteúdo textual. A "Imagem A" recebeu elogios por sua estética visual, enquanto a "Imagem B" se destacou por oferecer uma experiência mais intuitiva e funcional. As sugestões apontaram para a possibilidade de combinar as forças de ambas as imagens, aproveitando a estética da "Imagem A" com a funcionalidade da "Imagem B", ressaltando a importância de equilibrar forma e função na criação de interfaces de usuário.
    
    \item \textbf{Análise da Tela Principal}: Na avaliação das "Imagem A" e "Imagem B" para a tela principal, a preferência geral recai sobre a "Imagem B" em uma série de aspectos, incluindo o \textit{layout} atrativo, abordagem objetiva na disposição de elementos, experiência de uso confortável, organização eficiente de informações, distribuição equilibrada de elementos e clareza nos blocos de conteúdo. A "Imagem A" foi elogiada apenas por sua estética visual em algumas respostas. A sugestão apontada é a de que a "Imagem B" poderia ser aprimorada com cores mais claras nos elementos do corpo da página, semelhantes à "Imagem A". A predominância da preferência pela "Imagem B" sugere uma abordagem mais funcional e eficaz na criação da tela principal.
    
    \item \textbf{Análise da Tela de Conteúdo}: Na avaliação das "Imagem A" e "Imagem B" para a tela de conteúdo, a preferência geral recai sobre a "Imagem B" em diversos aspectos, incluindo o layout atrativo, abordagem objetiva na disposição e organização dos blocos de conteúdo, experiência de uso confortável, organização eficiente de informações e distribuição equilibrada de elementos. A sugestão apontada é a de que a "Imagem B" poderia incorporar um modo escuro (\textit{dark mode}) para maior conforto visual, especialmente ao exibir código na tela. A predominância da preferência pela "Imagem B" sugere que sua abordagem é mais eficaz e intuitiva para a apresentação de conteúdo.
\end{enumerate}

Em suma, os resultados da análise das telas de login, cadastro, tela principal e tela de conteúdo revelaram uma tendência clara em favor da "Imagem B" por parte dos estudantes que participaram do questionário. Essa preferência foi fundamentada em vários aspectos, como \textit{layout} atrativo, organização intuitiva, distribuição eficiente de informações e mensagens de erro adequadas. No entanto, é importante notar que a "Imagem A" também recebeu elogios por sua estética visual e pela presença de \textit{labels} no formulário, que foram considerados benéficos para usuários mais velhos.

As sugestões dos participantes apontaram para a possibilidade de combinar elementos positivos de ambas as imagens, destacando a importância de equilibrar forma e função na criação de interfaces de usuário. Além disso, houve recomendações para melhorias específicas, como a incorporação de um modo escuro na "Imagem B" para melhorar o conforto visual durante a visualização de código.

Em última análise, os resultados enfatizam a importância de considerar as preferências e necessidades dos usuários ao projetar interfaces, buscando o equilíbrio entre estética e funcionalidade para proporcionar uma experiência de usuário superior.



\section{Resultado - Pesquisa de Opinião com Profissionais}
\label{Pesquisa_de_Opinião_com_Profissionais}
A pesquisa foi conduzida com um grupo de participantes que demonstram uma forte ligação com o campo do desenvolvimento de software e áreas relacionadas. Estes respondentes apresentam uma variedade de perfis, incluindo diferentes faixas etárias e níveis de escolaridade, com muitos deles possuindo graduações e especializações.

A maioria dos participantes está atualmente trabalhando com desenvolvimento de software ou funções semelhantes, ocupando cargos que variam de estagiários a profissionais sêniores. Suas profissões abrangem várias áreas, como desenvolvedores \textit{full stack}, engenheiros de software, programadores, entre outros. Essa diversidade de cargos e níveis de experiência contribui para uma visão abrangente das perspectivas dos profissionais de desenvolvimento.

É notável que a maioria dos participantes possui conhecimento tanto em \ac{UI} quanto em \ac{UX}, o que demonstra a familiaridade desses profissionais com os aspectos cruciais do design de interfaces e da experiência do usuário.

A partir dessas informações, é possível esperar que as respostas e avaliações dos participantes forneçam \textit{insights} valiosos sobre as telas de interface do usuário e a experiência do usuário, com base em suas experiências e conhecimentos especializados. Essa perspectiva especializada é fundamental para uma análise aprofundada da pesquisa em questão. Vamos agora prosseguir com a análise das respostas desses participantes para obter percepções adicionais sobre os tópicos em discussão.

\begin{enumerate}
    \item \textbf{Análise da Tela de Login}: Durante a pesquisa, ao compararmos as telas de login "Imagem A" e "Imagem B", notamos uma clara preferência pela "Imagem B" entre os participantes. Ela foi considerada mais atrativa, intuitiva e com uma organização de informações superior. Além disso, a "Imagem B" se destacou por possuir uma distribuição mais eficiente dos campos de entrada e mensagens de erro adequadas. Apesar de algumas críticas pontuais à "Imagem A", como a presença de elementos visuais excessivos e possíveis problemas de legibilidade, a maioria dos participantes concordou que a "Imagem B" é a opção mais objetiva e eficaz para fazer login.
    
    \item \textbf{Análise da Tela de Cadastro}: As telas de cadastro também foram avaliadas, e novamente, a "Imagem B" se destacou. Ela foi consistentemente elogiada por sua organização, intuitividade e mensagens de erro adequadas. Por outro lado, a "Imagem A" muitas vezes foi vista como sobrecarregada de informações ou elementos visuais excessivos. No entanto, um ponto positivo da "Imagem A" foi a presença de \textit{labels} no formulário, que os participantes consideraram úteis, especialmente para os usuários mais velhos. Portanto, este estudo indica a preferência pela "Imagem B" para fins de cadastro, mas também ressalta a importância de considerar características específicas, como os \textit{labels}, para melhorar a experiência do usuário.
    
    \item \textbf{Análise da Tela Principal}: Ao avaliar as telas "Imagem A" e "Imagem B" da página principal, a "Imagem B" se destacou em aspectos cruciais, como \textit{layout}, organização e funcionalidade. Ela foi consistentemente escolhida pelos participantes como a mais eficaz em proporcionar uma experiência de usuário intuitiva e visualmente agradável. Embora a "Imagem A" tenha recebido algumas críticas, como a presença de informações excessivas, ainda foi elogiada por sua navegação em alguns casos. Em resumo, os resultados indicam uma clara preferência dos participantes pela "Imagem B" em design e usabilidade.

    \item \textbf{Análise da Tela de Conteúdo}: Ao analisar as respostas sobre a comparação entre as telas de conteúdo "Imagem A" e "Imagem B", percebemos que a maioria preferiu a "Imagem B" em design atrativo, organização, interatividade e consistência textual. A "Imagem B" se destacou principalmente por sua abordagem objetiva, distribuição equilibrada de elementos e eficiência na apresentação das informações. Por outro lado, a "Imagem A" foi ocasionalmente apontada como tendo excesso de informações ou elementos visuais, o que pode prejudicar a experiência do usuário. Em resumo, a maioria dos participantes identificou a "Imagem B" como a opção mais adequada para proporcionar uma experiência de usuário superior na apresentação de conteúdo.
    
\end{enumerate}

Em síntese, a análise abrangente das quatro áreas distintas, com base nas respostas de profissionais da área, revelou uma clara preferência pela "Imagem B" em quase todos os aspectos avaliados. Essa preferência foi fundamentada na atratividade visual, intuição, organização eficaz de informações e distribuição eficiente de elementos.

Apesar disso, a "Imagem A" não deve ser ignorada, pois, demonstrou méritos específicos, como a presença de \textit{labels} úteis para facilitar a experiência de usuários mais velhos. Portanto, a pesquisa indica a "Imagem B" como a escolha preferencial para a tela de login, cadastro, tela principal e tela de conteúdo. No entanto, destaca-se a importância de avaliar cuidadosamente as características específicas de cada imagem, aproveitando os pontos fortes de ambas, para otimizar a experiência do usuário de maneira global e eficaz.


%\section{Resultados}
%\label{Resultados}




